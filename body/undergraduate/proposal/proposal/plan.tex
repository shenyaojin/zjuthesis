\section{研究计划进度安排及预期目标}

\subsection{进度安排}

1. 2022.10.20-2022.11.30,文献调研,物性数据分析,完成物性数据初步分析。

2. 2022.12.1-2022.12.31,完成文献调研和撰写文献综述。完成地震与电法数据初步处理。完成外文翻译。完成开题报告撰写,包括研究目的与意义,国内外研究现状,论文研究内容,拟采用的技术路线及可行性分析,创新性分析,以及预期研究成果等。

3. 2023.1.1-2023.1.31,完成开题答辩。完成物性数据分析。完成地震与电法理论数据联合反演处理。

4. 2023.2.1-2023.2.28,完成地震与电法实际数据联合反演处理。准备提交开题报告、文献综述、外文翻译稿及原稿。

5. 2023.3.1-2023.3.31,完成论文初稿撰写。

6. 2023.4.1-2023.4.30,完成送审论文撰写。

7. 2023.5.1-2023.5.31,提交送审毕业论文。根据评审意见完成修改。完成毕业论文答辩。

8. 2023.6.1-2023.6.10,提交最终版毕业论文。

\subsection{预期目标}

在本研究中,我们希望实现浅地表电阻率和地震联合反演的算法,并且首先在广泛应用的嵌套模型和差异模型上测试单独反演、直接联合反演和加权联合反演的有效性,最后,将相应算法应用到浙江良渚古城墙周围探测的野外实例中。